\documentclass[12pt,a4paper]{article}

\usepackage{hyperref}

\parskip 0.3 cm
\parindent 0 cm

\begin{document}

\section{PBR Hackaton Projects}
\label{proj}

\subsection{Gasometer (Gas Flux)}
\label{gas}

Extend the exisiting setup (co2meter+arduino+screen): 

Sensor calibration routine via touch-screen (use PSI gas mixing system) 

Add Aalborg mass flow meter (arduino hardware serial Tx3,Rx3) and/or 

Valve control to measure several reactors (arduino software serial
connections), perhaps attach to PSI Multicultivator

\subsection{Spectrometer (Light Flux)} 
\label{spec}

Simple measuring tool: AvaSpec-Mini2048l-U25 + arduino or raspi

Advanced: with LED for absorbance, reflectance, or fluorescence,
built light paths and perhaps a reactor probe for online recording

\subsection{Continuous Culture (Liquid Flux)} 
\label{cult}

Balance and peristaltic pumps + arduino

Build simple reactor with gassing system measured by \ref{gas}

Perhaps combine with \ref{spec} to make turbidostatic control

\subsection{Microfluidics Device (Single Cell Biology)} 
\label{micro}

Scratch microscope slide + 2-3 pumps + arduino/screen

Connect to Ilka's lab microscope


\newpage

\section{Program}

\subsection{Day 1 $<$12:00 : Building Bioreactors}

Talks, 30-60 min:

\subsubsection{Rob's DIY Reactor - the beginnings}
\subsubsection{Dougie's DIY Reactor - 20 yrs later}
\subsubsection{Avantes - Spectrometry}

Spectrometry applications, incl. NIR for metabolite measurements and OD

Software interface to Avantes spectrometers

\subsubsection{CellDeg - Optimizing Photosynthetic Growth}

Introduction to CellDeg's 2.5 k algal growth setup (overnight 30 g/L
cyano biomass)

\subsection{Day 1 $>$13:00 : Hackathon I}

Introduction to the gasometer: connecting sensors with Arduino,
making an autonomous measurement device via Sainsmart's Touch Screen

Introduction to Rob's reactor: complete setup for photosynthetic growth

Self-organizing into teams: lab hardware (tubing etc.), control hardware
(soldering etc.), software

\subsection{Day 2 $<$12:00 : Photobioreactors in Research}

Talks, 30-60 min:

Nir Keren, Hellingwerf, Jan Cerveny, Dougie Murray,
something microfluidics?

\subsection{Day 2 $>$13:00 : Hackathon II}

Perhaps in teams, either by projects (\ref{gas}--\ref{micro}) or in
software vs. hardware (soldering/tubing) vs. biolab (cell cultures),
or -- most likely -- in dynamic self-organisation, working parallel on
all projects.

\subsubsection{Hardware I} 
soldering, tubing

\subsubsection{Software I} 
probe/sensor/pump $\Leftrightarrow$ arduino/raspi interfaces

\subsection{Day 3 $<$12:00 : Hackathon III}

\subsubsection{Hardware II} 
Visit HHU's fine mechanics and glas blower work-shops, place orders

Integrate projects \ref{gas},\ref{spec}\&\ref{cult} into a simple DIY
reactor and/or with PSI FMT150 or Multicultivator

Integrate project \ref{micro} with the simple microscope in Ilka's lab,
or a more advanced system (CAi?)

\subsection{Day 3 $>$13:00 : Consolidating}

\subsubsection{Software II}
arduino/raspi $\Leftrightarrow$  master/server interface

Standard formats and interfaces

Brain storming: relation of data and models

Beer: relation of data and models and beer


\end{document}
